\documentclass{report}

\usepackage[utf8]{inputenc}
\usepackage{tocloft}
\usepackage[french]{babel}
\usepackage{blindtext}
\usepackage{titlesec}
\usepackage{setspace}
\usepackage[normalem]{ulem}
\useunder{\uline}{\ul}{}
\usepackage{booktabs}

\author{sqrt(100)}
\title{Castle Of Demise}
\date{2023}

\renewcommand\cftchapaftersnum{.}
\renewcommand\cftsecaftersnum{.}


\renewcommand\thesection{\Roman{section}}
\renewcommand\thesubsection{(\arabic{subsection})}


\begin{document}

\maketitle

\doublespacing

\tableofcontents
\singlespacing

\newpage


\section{Introduction}


Ce document aura pour objectif de présenter le jeu Castle of Demise et son équipe de développement, sqrt(100). Il vise à offrir un aperçu global de notre vision du projet et des principaux points du cahier des charges.


Le projet Castle of Demise trouve ses racines dans la passion des membres du groupe pour les jeux vidéo, en particulier les FPS (First Person Shooter). Inspiré par des classiques tels que Doom, ce type de jeux se caractérise par un style graphique simple et rétro, mettant l'accent sur des textures pixelisées pour une expérience utilisateur simple et agréable. \newline 


L’intérêt du projet est multiple : Tout d'abord, le développement de Castle of Demise permettra à chaque membre du groupe de diversifier ses compétences techniques (en se spécialisant dans le graphisme, la modélisation 3D, le level-design, la communication en réseau (LAN) et même le langage C\#). \newline


Les efforts mis en œuvre permettront également d’appliquer les méthodes théoriques apprises durant les cours à l’Epita : toute l’organisation de ce projet est basée sur les concepts assimilés durant nos séances de méthodologies telles que la planification, la méthode SMART, ou même la communication interpersonnelle dans le cadre des échanges d’informations pour la mise en commun des recherches. \newline

Le développement de ce jeu consolidera également les notions techniques de nos cours de sciences : la récursivité, la programmation orientée-objet (POO) et le langage C\# en général sont au cœur du projet. 


De plus, au-delà des avantages propres à notre groupe, les joueurs de Castle of Demise bénéficieront d’une expérience de jeu unique et personnalisée. C’est autour de cette idée fondamentale que se base tout le développement du jeu : par l’intégration de modèles linguistiques d’Intelligence artificielle, l’effet que nous recherchons est que le joueur se sente véritablement comme l’élément central du jeu, qu’il s’identifie au personnage qu’il incarne. En outre, il faut que le jeu s’adresse directement au joueur. \newline


Cette ergonomie est également assurée par un gameplay agréable et fluide : un système de niveaux de jeu ajustable permettra une accessibilité accrue pour les débutants, afin de toucher le plus grand nombre de joueurs. 
Enfin, l’idée d’optimisation logicielle sera très mise en avant dans le cadre du développement de notre jeu, pour rendre son exécution fluide sur le plus grand nombre de terminaux possibles. Disponible sur Windows, Castle of Demise se veut donc accessible au plus grand nombre par sa facilité de prise en main et sa compatibilité. \newline

\newpage
\section{Origine Du Projet}

L’idée de Castle of Demise puise ses origines dans la passion commune du groupe: le jeu vidéo. Baignés depuis notre jeunesse dans la culture du FPS (First Person Shooter), l’idée de créer un jeu dans ce style universel nous est parue comme une évidence. \newline

Le style graphique du jeu est flatteur et rétro : caractérisé par des textures visuelles de basse qualité (pixélisées) et un éclairage simpliste, Castle of Demise permet à notre équipe de créer une interface simple mais soignée, et de se focaliser sur l’essentiel : l’expérience utilisateur finale.


\section{Objectifs et intérêts}
Castle of Demise engendre une multiplicité d’intérêts et d’objectifs, aussi bien sur le plan collectif qu’individuel, dans le contexte du développement collaboratif au sein de l’équipe de conception. \newline 

 En premier lieu, cette initiative génère un environnement d’organisation rigoureux, réfléchi et structuré, qui force chacun des membres de notre équipe à s’adapter aux conditions de travail communes. Ces conditions, par leur similitude avec celles prévalant dans le monde professionnel, offrent une opportunité unique d’acquérir des compétences organisationnelles essentielles, en plus d’assurer une synergie efficace et harmonieuse entre les collaborateurs. \newline
 
Sur le plan individuel, le projet Castle of Demise revêt une importance capitale, car il permet à chaque membre du groupe de partager leurs connaissances et compétences acquises en dehors du cadre scolaire. Par exemple, Paul Lacanette a fait preuve d’une expertise remarquable dans l’utilisation du logiciel Godot, une compétence précieuse qui a été mise à profit pour le développement du jeu. \newline 

Cette collaboration entre les membres de l’équipe, où chacun apporte sa contribution unique, se traduit non seulement par une avancée significative dans le processus de développement du jeu, mais renforce également l’esprit d’entraide au sein du groupe.
Cette dynamique d’entraide s’adapte aux compétences et aux défis individuels, favorisant ainsi un environnement propice à notre croissance personnelle et collective. \newline


Par ailleurs, lors de la phase initiale de réflexion et de conception du jeu, l’approche de soumettre des idées inspirées de jeux préexistants s’est révélée extrêmement enrichissante. En effet, cette démarche a permis à notre équipe de se confronter à des contraintes techniques complexes qui requéraient des solutions innovantes et personnalisées, stimulant ainsi la recherche et le développement de compétences en dehors du socle commun de la préparation.
La mise en œuvre de certaines fonctionnalités du jeu a nécessité l’acquisition de compétences dans des outils parallèles à ceux généralement enseignés, élargissant ainsi notre spectre de connaissances et renforçant notre capacité à relever des défis technologiques variés. \newline

Le projet Castle of Demise incarnera donc une opportunité unique d’acquérir des compétences professionnelles en matière d’organisation, de partager des compétences individuelles exceptionnelles, et de repousser les limites de notre expertise technique, tout en favorisant une collaboration solide et une expérience de développement de jeu enrichissante pour l’ensemble de l’équipe.

\section{État de l'Art}

Le développement de Castle of Demise s’inspire profondément des pionniers du genre du FPS, notamment des œuvres emblématiques produites par id Software, une société fondée par John Romero, Tom Hall, John et Adrian Carmack. \newline 

Cette entreprise a marqué l’histoire du jeu vidéo en donnant naissance à des titres phares tels que Wolfenstein 3D, Doom et Quake, qui ont eu une influence durable sur le genre des FPS, à tel point que pendant un certain temps, les jeux de ce type étaient communément désignés sous le terme de "Doom-likes", dont nous nous sommes inspirés de l’appellation pour notre jeu. \newline

Wolfenstein 3D, sorti en 1992, est le premier FPS de l’histoire. Il se distingue non seulement par la première utilisation d’un moteur de rendu 3D, mais aussi par sa capacité à immerger les joueurs dans un environnement tridimensionnel grâce à la vue à la
première personne. Cela a marqué une véritable révolution dans le monde du jeu vidéo,
et Wolfenstein 3D a joué un rôle crucial dans la démocratisation du genre FPS. \newline

Sur le plan technique, le jeu a été novateur à l’époque, car il a utilisé des textures, des animations et des sprites en 2D, tandis que le décor était modélisé en 3D, une stratégie qui a également été mise en œuvre dans Doom. \newline

Cette approche a permis une optimisation et une adaptabilité du jeu à une variété de plateformes, un défi que la communauté continue de relever en portant (c’est à dire en développant) Doom sur divers supports. \newline

Doom, sorti en 1993, a poursuivi le travail technique initié par Wolfenstein 3D. Il a amélioré le level design en ajoutant plus de verticalité, offrant des environnements, des décors et un bestiaire plus soigné et varié que son prédécesseur. Le moteur de Doom fut si complet et polyvalent qu’il est toujours utilisé par la communauté pour créer de nouveaux jeux et des mods, témoignant de sa longévité. On peut citer Selaco et Prodeus, deux jeux récents se basant pourtant sur ce moteur. \newline

Doom a rencontré un succès commercial massif, avec 3,5 millions de copies vendues entre 1993 et 1999, ainsi qu’une estimation de 20 millions de joueurs sur la même période, basée sur le nombre de téléchargements du premier épisode gratuit du jeu. Ce jeu a grandement contribué à la popularisation des FPS, et sa stratégie d’utiliser principalement des éléments en 2D a également inspiré la conception de Castle of Demise. \newline

Quake, sorti après Doom, a apporté une révolution technique majeure en offrant
un moteur qui, contrairement à celui de son prédécesseur, était capable de produire un rendu 3D en temps réel. Quake a également introduit la possibilité de sauter et de regarder de haut en bas, ajoutant ainsi plus de verticalité au gameplay. Ce jeu a également été l’un
des premiers FPS de l’histoire à proposer un mode multijoueur sous la forme d’un Deathmatch jouable soit en ligne soit en LAN, fonctionnalité ayant grandement inspiré le mode multijoueur de notre jeu, dont le LAN sera aussi l’interface primaire. \newline

D’autres jeux nous ont inspirés sur le côté artistique de Castle of Demise. Ultrakill et Castlevania font partie de nos plus grandes sources d’inspirations. (Mention honorable à Devil May Cry premier du nom, dont l’architecture est, par nature, très inspirante). \newline

Ultrakill, un jeu de type Fast FPS, privilégiant ainsi la mobilité, est disponible en accès anticipé depuis 2020 et se déroule dans un univers postapocalyptique. Ce jeu se distingue par sa retenue manifeste dans le domaine des interactions verbales, les échanges dialogués étant d’une rareté épisodique, ce qui contribue à instaurer une atmosphère de
jeu où la concision communicationnelle prévaut. \newline

 Le joueur incarne un robot doté de la faculté d’absorber le sang de ses adversaires pour régénérer sa propre vitalité. La narration s’articule principalement à travers l’environnement du jeu, des éléments scripturaux judicieusement disséminés dans les niveaux, et des conversations sporadiques avec les
rares personnages présents. Cette approche narrative, conjuguée aux caractéristiques esthétiques des protagonistes, a constitué une source d’inspiration significative pour
Castle of Demise. \newline

Enfin, La série de jeux Castlevania se distingue comme une pionnière dans la création d’un genre vidéoludique bien distinct, communément désigné sous le terme "metroidvania".
Il s’agit d’une saga qui a marqué un jalon essentiel dans l’histoire du jeu vidéo, en introduisant l’horreur gothique dans un contexte interactif. Cette franchise s’est particulièrement démarquée en s’inspirant de manière profonde de l’univers du roman classique "Dracula" de Bram Stoker, allant jusqu’à emprunter des éléments substantiels de l’intrigue, des personnages, et de l’atmosphère de cet ouvrage emblématique. \newline

Toutefois, c’est l’univers immersif et le bestiaire emblématique de la série "Castlevania" qui ont constitué une source d’inspiration majeure pour la conception de notre propre jeu vidéo. \newline

L’ensemble de ces jeux a posé les bases techniques et artistiques pour le développement de Castle of Demise, en visant à offrir une expérience de jeu soignée, qualitative et ambitieuse, tout en rendant hommage à l’héritage des pionniers du genre FPS ainsi qu’aux jeux qu’à d’autres jeux qu’on tient particulièrement à cœur. \newline



\section{Origine de l'entreprise}

Le nom de notre entreprise, sqrt(100), a été consciencieusement choisi en quête
de simplicité et de clarté. Dans le domaine des mathématiques, la racine carrée de 100 équivaut à un nombre rond et est souvent associée à la perfection : 10. \newline

 Ce chiffre symbolique incarne notre ferme engagement à simplifier des énigmes complexes pour le produit final, tout comme la racine carrée simplifie la recherche de la valeur d’un nombre, évoquant la précision et l’excellence. En somme, sqrt(100) est un emblème de clarté, de simplicité, et un gage de précision au cœur même des solutions que notre entreprise propose. \newline

Distinguer notre entreprise, c’est avant tout souligner son caractère unique. Nous sommes une entreprise à but non lucratif, ce qui signifie que notre principal objectif ne réside pas dans la quête effrénée de revenus tangibles. Notre mission première consiste à libérer notre créativité pour donner naissance à des jeux originaux qui apportent de la joie à ceux qui les découvrent. \newline

 Notre spécialisation réside dans le domaine du développement de jeux vidéo, car nous croyons en la nécessité de nous démarquer dans un secteur où l’excellence est cruciale. \newline

Ce qui nous distingue encore davantage est notre approche en matière de contenus. Tous les éléments qui composent nos produits, qu’il s’agisse d’images, de textures, de musiques ou de l’univers narratif du jeu, sont le fruit de notre création ou d’une collaboration artistique étroite. Aucun de ces éléments n’est acquis auprès de sources extérieures ou acheté en ligne, renforçant ainsi notre engagement envers l’authenticité et la qualité de nos créations.


\section{Présentation des membres}

	\subsection{\normalsize Amadéo HÉAULME}
	
	Amadéo, en tant que membre clé de l’équipe de développement du jeu, se
	distingue par son vif intérêt pour l’intelligence artificielle. Son rôle prépondérant est
	la programmation de l’IA des ennemis, une tâche cruciale pour garantir une expérience de jeu immersive et stimulante. Par ailleurs, Amadéo apportera une touche ludique au projet en insérant des artefacts "easter eggs", ajoutant des éléments surprenants et divertissants pour les joueurs. \newline
	
	 Il se chargera également de l’enrichissement du jeu en proposant des fonctionnalités additionnelles. En tant que gestionnaire du dépôt Github, Amadéo veillera à ce que toutes les mises à jour soient bien documentées et communiquent efficacement avec l’équipe via Discord, garantissant ainsi une collaboration fluide et transparente.
	 
	 
	\subsection{\normalsize Jean HERAIL}
		Jean assume le rôle de chef d’équipe au sein du projet. Doté d’une solide expérience en programmation et d’une familiarité avec l’intelligence artificielle, il apporte une expertise technique essentielle pour la réalisation du jeu. \newline 
		
		De plus, Jean démontre un intérêt pour la musique, ce qui pourrait potentiellement enrichir l’aspect sonore du jeu. Son expérience passée dans la manipulation des modèles GPT d’OpenAI et des APIs correspondantes est un atout pour résoudre des problèmes techniques spécifiques. \newline
		
		Ayant des bases de "prompt-engineering", il se chargera de l’intégration de l’IA dans le jeu, créant alors une interface OpenAI / CoD (Castle of Demise). En tant que chef d’équipe, il est en charge de coordonner les activités de l’équipe. Il assistera également Paul dans la réalisation de la partie LAN (multijoueur).
		
	\subsection{\normalsize Roni YILDIZ}
	Roni, le consultant culturel vidéoludique, apporte une richesse de connaissances
	et une passion profonde pour les jeux vidéo au sein de l’équipe. Fortement imprégné de
	la culture du jeu vidéo depuis son plus jeune âge, il offre des références précieuses pour
	le projet Castle of Demise. \newline 
	
	Son rôle consiste également à développer l’aspect créatif du jeu, notamment en élaborant l’univers, le lore (univers virtuel dans lequel se passe le jeu), et en supervisant la modélisation des éléments du jeu. La contribution de Roni assure une profonde immersion dans le monde du jeu et garantit une expérience captivante pour les joueurs.

	\subsection{\normalsize Clément GRONDIN}
	Clément, bien qu’il possède déjà une solide connaissance de la culture vidéoludique, apporte une perspective exceptionnellement rafraîchissante au développement du jeu. En qualité de membre de notre équipe, sa curiosité insatiable et son enthousiasme pour l’exploration de nouvelles voies créatives se révèlent être de précieux atouts. \newline
	
	 Clément démontre un engagement assidu à l’acquisition de nouvelles compétences, et cette volonté d’en apprendre s’accompagne d’une approche novatrice inestimable au projet. Sa participation active est source d’inspiration pour l’ensemble de l’équipe, contribuant de manière significative à l’évolution positive de notre projet. Sa propension à la collaboration et à la motivation renforce l’unité et la synergie du collectif, garantissant ainsi un environnement propice à la réussite.
	 
	 
	
	\subsection{\normalsize Paul LACANETTE}
	Paul, quant à lui, combine une solide culture vidéoludique avec un intérêt prononcé pour la musique et le développement de jeux vidéo. Son expérience préalable avec Godot, ainsi que sa passion pour les jeux de tir à la première personne, font de lui un atout majeur pour le projet. \newline
	
	Il se consacre à l’approfondissement de ses compétences et à la réalisation d’un jeu vidéo de qualité, faisant de Castle of Demise une opportunité idéale pour concrétiser ses aspirations dans le domaine du développement de jeux.


	
	
	
\section{Assignation des tâches}


\begin{center}
\begin{tabular}{@{}ll@{}}
\toprule
\textbf{Tâche}               & \textbf{Responsable / Suppléant} \\ \midrule
Level-Design                 & Paul / Roni                       \\
Modélisation graphique      & Roni / Paul                       \\
Intégration de l’IA         & Jean / Amadéo                    \\
Prise en charge du réseau   & Amadéo / Jean                    \\
Jouabilité                   & Paul / Clément                   \\
Développement du site Web   & Clément / Roni                   \\
Élaboration de la bande-son & Jean / Clément                   \\
Chef du groupe               & Jean                              \\
Lore                         & Roni / Amadéo                    \\ \bottomrule
\end{tabular}
\end{center}

\newpage


\section{Aspect économique}
Dans l’examen minutieux de l’aspect budgétaire et économique de notre projet, il
devient impératif de plonger dans la profondeur de la gestion des coûts associés à
l’utilisation de l’API de GPT-3. \newline

Cette interface de programmation, conçue par OpenAI,
repose sur un mécanisme de paiement basé sur l’unité de "tokens." Il est crucial de noter
que 1000 tokens, selon les directives fournies par le site officiel d’OpenAI, équivalent
approximativement à 750 mots, ce qui établit un lien direct entre l’utilisation de cette
puissante API et le volume de texte qu’elle génère. \newline

Une composante essentielle de ce défi budgétaire réside dans la variété de modèles
offerts par l’API, parmi lesquels émerge "text-davinci-003" comme le choix retenu pour
notre projet. Ce modèle, en particulier, est assujetti à une tarification qui mérite une
attention particulière. \newline

En effet, chaque tranche de 1000 tokens se voit affectée d’un coût
de \$0,002, ce qui constitue le socle sur lequel s’érige notre approche économique.
Il est impératif que nous prenions conscience de l’impact potentiel sur notre
budget, tout en reconnaissant la flexibilité inhérente à cette API qui nous permet d’adapter
notre utilisation en fonction de nos besoins spécifiques. \newline

Cela exige une gestion minutieuse des ressources et une compréhension approfondie des mécanismes tarifaires associés à chaque modèle, tout en maintenant une vigilance constante sur l’optimisation de notre utilisation pour garantir une utilisation judicieuse de ces ressources financières. \newline

En somme, le volet budgétaire et économique de notre projet, étroitement lié à l’exploitation de l’API GPT-3, nécessite une attention rigoureuse aux détails, une compréhension approfondie des mécanismes de tarification, et une gestion astucieuse de nos ressources pour maintenir notre projet sur la voie de la réussite, tout en garantissant un usage efficient de cette technologie.

\newpage

\section{Conclusion}
Le projet Castle of Demise est donc un projet qui nous tient profondément à cœur. Non seulement parce qu’il représente une opportunité scolaire très pertinente, mais aussi parce qu’il résonne avec nos aspirations individuelles et collectives. Pour chacun d’entre nous, ce projet est une occasion de croissance et d’apprentissage. \newline

Il est intrinsèquement intéressant, car il s’inscrit dans notre passion commune pour les jeux vidéo, et plus particulièrement pour le genre FPS. Castle of Demise nous permet d’explorer et de concrétiser cette passion, en créant un jeu qui s’inspire des classiques tels que Doom. Le caractère rétro et la simplicité graphique du jeu ajoutent une dimension nostalgique qui nous enchante. \newline

Paul Lacannette se spécialisera dans le Level Design, un domaine essentiel pour créer une expérience de jeu immersive. Il sera en charge de créer des niveaux passionnants et de donner vie à l’univers du jeu. Il perfectionnera avec Roni Yildiz ses compétences en modélisation graphique, contribuant apporter au jeu une identité visuelle unique. Ses créations donneront vie aux personnages et environnements. \newline

Roni Yildiz jouera un rôle de Consultant Culturel Vidéoludique, apportant des références et des idées inspirantes qui enrichiront le gameplay et l’immersion. Clément Grondin se plongera dans le développement du site Web, maîtrisant les outils nécessaires pour promouvoir le jeu, partager des informations et maintenir une communication efficace. \newline

Amadéo Héaulme se spécialisera avec Jean Herail dans l’IA, en travaillant sur les comportements des ennemis et la personnalisation de l’expérience du joueur. Leur travail nous permettra de créer des adversaires intelligents et captivants. \newline

Enfin, Jean Herail prendra responsabilité de coordonner les activités de l’équipe. Son expérience en programmation et sa familiarité avec l’intelligence artificielle sont des atouts majeurs pour résoudre des problèmes techniques spécifiques, tandis que sa maîtrise des modèles GPT d’OpenAI et des APIs correspondantes renforce notre capacité à innover. \newline

Castle of Demise est bien plus qu’un simple projet, c’est une aventure qui stimule notre curiosité, renforce nos compétences et nourrit notre passion. Nous sommes impatients de voir ce projet évoluer, avec la certitude qu’il façonnera notre avenir en tant qu’étudiants à l’Epita.





\end{document}
